\chapter{SOC Preparation}

\section*{Security Operations Center (SOC) Functions – PowerPlus}

\subsection*{ Log Collection}
\textbf{Description:} \\
PowerPlus must implement centralized log collection from all critical systems, including servers, databases, IT/OT applications, network devices, and BYOD endpoints. \\

\textbf{Objective:} \\
Enable real-time monitoring of security events, providing a comprehensive view of activities across the infrastructure. \\

\textbf{Recommendations:}
\begin{itemize}
    \item Integrate 200 applications and OT devices with the SIEM solution for log collection.
    \item Capture critical logs related to remote access by external vendors responsible for OT system maintenance.
\end{itemize}

\subsection*{ Aggregation/Correlation}
\textbf{Description:} \\
PowerPlus must correlate security events from different sources to identify patterns that may indicate threats. \\

\textbf{Objective:} \\
Detect complex attacks that might go unnoticed in isolated analyses. \\

\textbf{Recommendations:}
\begin{itemize}
    \item Configure correlation rules to monitor unauthorized access between IT and OT systems, preventing lateral movement between critical environments.
    \item Monitor the 20 applications handling personal data of 6 million customers for suspicious activities.
\end{itemize}

\subsection*{ SIEM (Security Information and Event Management)}
\textbf{Description:} \\
PowerPlus already utilizes a SIEM platform for security event integration and correlation. The SIEM will be the core of the SOC, aggregating, analyzing, and alerting on potential incidents. \\

\textbf{Objective:} \\
Provide a centralized view of security operations and detect threats in real-time. \\

\textbf{Recommendations:}
\begin{itemize}
    \item Ensure that the SIEM is configured to integrate logs from 1,500 production servers, as well as critical cloud applications (SaaS and IaaS).
    \item Regularly review and update correlation rules in the SIEM based on new attack scenarios.
\end{itemize}

\subsection*{ Threat Intelligence}
\textbf{Description:} \\
PowerPlus should integrate threat intelligence feeds into the SOC to anticipate new attacks and vulnerabilities. \\

\textbf{Objective:} \\
Enhance the SOC's proactive capability to prevent incidents before they occur. \\

\textbf{Recommendations:}
\begin{itemize}
    \item Subscribe to global and industry-specific threat intelligence feeds, especially for the energy sector.
    \item Monitor vulnerabilities specific to Oracle and SAP technologies, widely used in PowerPlus systems.
\end{itemize}

\subsection*{ Research \& Development (R\&D)}
\textbf{Description:} \\
The SOC should invest in developing new tools, techniques, and strategies to address emerging threats. \\

\textbf{Objective:} \\
Keep the SOC updated and prepared to face new security challenges. \\

\textbf{Recommendations:}
\begin{itemize}
    \item Develop automated scripts for security monitoring in hybrid environments (on-premises and cloud).
    \item Conduct regular penetration tests on legacy SAP systems to identify and mitigate vulnerabilities.
\end{itemize}

\subsection*{Ticketing}
\textbf{Description:} \\
The SOC must use an incident management system to track, prioritize, and resolve security incidents. \\

\textbf{Objective:} \\
Ensure that each incident is appropriately handled and resolved within defined timelines. \\

\textbf{Recommendations:}
\begin{itemize}
    \item Implement a ticketing system integrated with the SIEM, with workflows prioritizing critical incidents in systems handling personal data and OT operations.
    \item Log all incidents involving remote access by vendors.
\end{itemize}

\subsection*{ Knowledge Base}
\textbf{Description:} \\
The SOC should maintain a repository of information on past incidents, response procedures, and best practices. \\

\textbf{Objective:} \\
Accelerate response to future incidents by reusing previously tested solutions. \\

\textbf{Recommendations:}
\begin{itemize}
    \item Document critical incidents involving OT and IT systems, detailing mitigation measures.
    \item Create a repository of response procedures for ransomware attacks, given the criticality of energy systems.
\end{itemize}

\subsection*{ Reporting}
\textbf{Description:} \\
The SOC should generate periodic reports on the organization's security posture and the performance of security operations. \\

\textbf{Objective:} \\
Provide insights to senior management on PowerPlus’s security posture and areas for improvement. \\

\textbf{Recommendations:}
\begin{itemize}
    \item Develop monthly reports detailing critical incidents, unauthorized access attempts, and detected vulnerabilities.
    \item Present key performance indicators (KPIs), such as Mean Time to Detect (MTTD) and Mean Time to Respond (MTTR).
\end{itemize}

\section*{Conclusion}
The SOC at PowerPlus should operate as a strategic unit responsible for protecting the organization's critical infrastructure against cyber threats. By implementing the functions described, the SOC will be capable of:
\begin{itemize}
    \item Monitoring and correlating security events in real-time.
    \item Responding quickly and effectively to critical incidents.
    \item Anticipating new threats based on threat intelligence and continuous research.
    \item Providing detailed reports to support strategic security decisions.
\end{itemize}
This approach will ensure that PowerPlus maintains secure operations, minimizing risks to customers, employees, and critical assets.

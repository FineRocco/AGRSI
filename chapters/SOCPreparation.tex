\chapter{SOC Preparation}

\section*{Security Operations Center (SOC) Functions – PowerPlus}

\section*{1. Aspects to be Monitored by the SOC}

\subsection*{ Log Collection}
\textbf{Description:} \\
PowerPlus must implement centralized log collection from all critical systems, including servers, databases, IT/OT applications, network devices, and BYOD endpoints. \\

\textbf{Objective:} \\
Enable real-time monitoring of security events, providing a comprehensive view of activities across the infrastructure. \\

\textbf{Recommendations:}
\begin{itemize}
    \item Integrate 200 applications and OT devices with the SIEM solution for log collection.
    \item Capture critical logs related to remote access by external vendors responsible for OT system maintenance.
\end{itemize}

\subsection*{ Aggregation/Correlation}
\textbf{Description:} \\
PowerPlus must correlate security events from different sources to identify patterns that may indicate threats. \\

\textbf{Objective:} \\
Detect complex attacks that might go unnoticed in isolated analyses. \\

\textbf{Recommendations:}
\begin{itemize}
    \item Configure correlation rules to monitor unauthorized access between IT and OT systems, preventing lateral movement between critical environments.
    \item Monitor the 20 applications handling personal data of 6 million customers for suspicious activities.
\end{itemize}

\subsection*{ SIEM (Security Information and Event Management)}
\textbf{Description:} \\
PowerPlus already utilizes a SIEM platform for security event integration and correlation. The SIEM will be the core of the SOC, aggregating, analyzing, and alerting on potential incidents. \\

\textbf{Objective:} \\
Provide a centralized view of security operations and detect threats in real-time. \\

\textbf{Recommendations:}
\begin{itemize}
    \item Ensure that the SIEM is configured to integrate logs from 1,500 production servers, as well as critical cloud applications (SaaS and IaaS).
    \item Regularly review and update correlation rules in the SIEM based on new attack scenarios.
\end{itemize}

\subsection*{ Threat Intelligence}
\textbf{Description:} \\
PowerPlus should integrate threat intelligence feeds into the SOC to anticipate new attacks and vulnerabilities. \\

\textbf{Objective:} \\
Enhance the SOC's proactive capability to prevent incidents before they occur. \\

\textbf{Recommendations:}
\begin{itemize}
    \item Subscribe to global and industry-specific threat intelligence feeds, especially for the energy sector.
    \item Monitor vulnerabilities specific to Oracle and SAP technologies, widely used in PowerPlus systems.
\end{itemize}

\subsection*{ Research \& Development (R\&D)}
\textbf{Description:} \\
The SOC should invest in developing new tools, techniques, and strategies to address emerging threats. \\

\textbf{Objective:} \\
Keep the SOC updated and prepared to face new security challenges. \\

\textbf{Recommendations:}
\begin{itemize}
    \item Develop automated scripts for security monitoring in hybrid environments (on-premises and cloud).
    \item Conduct regular penetration tests on legacy SAP systems to identify and mitigate vulnerabilities.
\end{itemize}

\subsection*{Ticketing}
\textbf{Description:} \\
The SOC must use an incident management system to track, prioritize, and resolve security incidents. \\

\textbf{Objective:} \\
Ensure that each incident is appropriately handled and resolved within defined timelines. \\

\textbf{Recommendations:}
\begin{itemize}
    \item Implement a ticketing system integrated with the SIEM, with workflows prioritizing critical incidents in systems handling personal data and OT operations.
    \item Log all incidents involving remote access by vendors.
\end{itemize}

\subsection*{ Knowledge Base}
\textbf{Description:} \\
The SOC should maintain a repository of information on past incidents, response procedures, and best practices. \\

\textbf{Objective:} \\
Accelerate response to future incidents by reusing previously tested solutions. \\

\textbf{Recommendations:}
\begin{itemize}
    \item Document critical incidents involving OT and IT systems, detailing mitigation measures.
    \item Create a repository of response procedures for ransomware attacks, given the criticality of energy systems.
\end{itemize}

\subsection*{ Reporting}
\textbf{Description:} \\
The SOC should generate periodic reports on the organization's security posture and the performance of security operations. \\

\textbf{Objective:} \\
Provide insights to senior management on PowerPlus’s security posture and areas for improvement. \\

\textbf{Recommendations:}
\begin{itemize}
    \item Develop monthly reports detailing critical incidents, unauthorized access attempts, and detected vulnerabilities.
    \item Present key performance indicators (KPIs), such as Mean Time to Detect (MTTD) and Mean Time to Respond (MTTR).
\end{itemize}

\section*{2. Incident Response Cases}

This section presents three detailed incident scenarios focusing on the response actions taken by the SOC.

\subsection{Incident Response Case 1: Ransomware Attack on IT Infrastructure}

\textbf{Description:}  
A ransomware attack encrypts critical IT systems, including production servers and databases that store customers' personal data.

\textbf{Impact:}  
\begin{itemize}
    \item Unavailability of billing and CRM systems.
    \item Potential leakage of personal data from 6 million customers.
\end{itemize}

\textbf{Response Procedure:}
\begin{enumerate}
    \item Initial detection by the SIEM, which identified anomalous activities on the production servers.
    \item Immediate isolation of the affected systems to prevent the ransomware from spreading.
    \item Execution of backups to restore critical systems.
    \item Communication with cloud providers to ensure the security of cloud environments.
\end{enumerate}

\textbf{Lessons Learned:}
\begin{itemize}
    \item Implement stricter network segmentation.
    \item Regularly update backup policies.
\end{itemize}

\subsection{Incident Response Case 2: Phishing Attack Targeting Employees}

\textbf{Description:}  
A phishing campaign targets Call Center employees to steal credentials for accessing internal systems.

\textbf{Impact:}  
\begin{itemize}
    \item Unauthorized access to customers' personal data.
    \item Potential compromise of administrative accounts.
\end{itemize}

\textbf{Response Procedure:}
\begin{enumerate}
    \item SIEM detects suspicious logins from unknown IP addresses.
    \item Immediate blocking of compromised accounts and password resets.
    \item Notification and training of employees about the phishing campaign.
    \item Review of SIEM correlation rules to detect similar patterns in the future.
\end{enumerate}

\textbf{Lessons Learned:}
\begin{itemize}
    \item Implement Multi-Factor Authentication (MFA) for all critical systems.
    \item Increase the frequency of security awareness training.
\end{itemize}

\subsection{Incident Response Case 3: Remote Access Breach on OT Systems}

\textbf{Description:}  
Unauthorized remote access to OT systems through compromised credentials of third-party vendors.

\textbf{Impact:}  
\begin{itemize}
    \item Risk of compromise to energy control systems.
    \item Potential disruption in the energy supply to customers.
\end{itemize}

\textbf{Response Procedure:}
\begin{enumerate}
    \item SIEM detects repeated login attempts outside vendors' working hours.
    \item Blocking of remote access and notification to OT system managers.
    \item Audit of access logs to identify potential compromises.
    \item Implementation of secure remote access solutions (VPNs and zero trust policies).
\end{enumerate}

\textbf{Lessons Learned:}
\begin{itemize}
    \item Improve network segmentation between IT and OT environments.
    \item Establish stricter remote access policies for third-party vendors.
\end{itemize}

\section*{Conclusion}

Establishing an efficient Security Operations Center (SOC) is crucial for PowerPlus to address cybersecurity risks. Centralized log collection, event correlation, threat intelligence, and continuous innovation are key to detecting and mitigating security incidents. The incident response cases highlight the importance of swift actions and adopting best practices such as network segmentation and employee training. With a robust SOC, PowerPlus will be better equipped to protect its infrastructure and data, ensuring the continuity of critical operations.





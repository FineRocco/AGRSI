\chapter{Introduction}

\section{Purpose of the Report}

This report aims to conduct a comprehensive risk assessment and develop an initial incident response plan for PowerPlus. The project addresses the organization's need to protect its critical assets, secure its technological infrastructure, and manage cybersecurity risks. The findings will guide the Information Security and IT Risk Management teams in implementing effective controls and preparedness strategies for potential security incidents.

\section{Background on PowerPlus}

%Provide an overview of PowerPlus as an organization, including its primary business operations, industry, and any relevant context that may influence its security needs. Highlight any notable regulatory, legal, or compliance requirements that impact PowerPlus’s information security approach. 
%This can include compliance with industry standards, data privacy laws (such as GDPR if applicable), and any sector-specific requirements.

\subsection{Industry-Specific Security Challenges}

%Briefly discuss common security challenges within PowerPlus’s industry, including frequent cyber threats or unique vulnerabilities that similar companies face. This section helps frame the risk assessment within an industry-relevant context, underscoring the necessity for strong cybersecurity measures.

\section{Objectives of the Risk Assessment and Incident Response Plan}

% List the primary objectives of the risk assessment and incident response plan, focusing on:

% Identifying, analyzing, and evaluating cybersecurity risks to PowerPlus.
% Developing a proactive response strategy for potential security incidents.
% Ensuring alignment with ISO/IEC 27005:2022 standards for systematic and effective risk management.
% Establishing guidelines for PowerPlus’s Security Operations Center (SOC) to detect, respond to, and mitigate incidents.
% These objectives reflect PowerPlus’s commitment to safeguarding its information assets and maintaining business continuity.

\section{Methodology Overview}

% Provide a brief summary of the methodology that will be used in the report. This includes:

% Utilizing ISO/IEC 27005:2022 as the primary standard for risk assessment, with references to ISO/IEC 27001:2022 and ISO 31000:2018 for additional security and risk management guidance.
% Employing a structured approach that involves risk identification, analysis, evaluation, and treatment, as outlined in the standards.
% Developing incident scenarios and SOC preparation strategies based on observed threats and vulnerabilities.

\section{Structure of the Report}

% Summarize the key chapters that follow in this report:

% Establishing the Context: Outlines the scope, assets, risk environment, and methodology applied in the assessment.
% Risk Assessment: Details the process and results of identifying and evaluating risks specific to PowerPlus.
% Incident Scenarios: Conceptualizes three hypothetical attack scenarios to assess PowerPlus’s readiness.
% SOC Preparation: Describes monitoring requirements and three detailed incident response cases for PowerPlus’s SOC.
% Conclusions: Summarizes key findings and recommendations based on the assessment.
% References: Lists all resources and standards referenced throughout the report in APA format.
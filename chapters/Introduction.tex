\chapter{Introduction}

\section{Purpose of the Report}

This report aims to conduct a comprehensive risk assessment and develop an initial incident response plan for PowerPlus. The project addresses the organization's need to protect its critical assets, secure its technological infrastructure, and manage cybersecurity risks. The findings will guide the Information Security and IT Risk Management teams in implementing effective controls and preparedness strategies for potential security incidents.

\section{Background on PowerPlus}

%Provide an overview of PowerPlus as an organization, including its primary business operations, industry, and any relevant context that may influence its security needs. Highlight any notable regulatory, legal, or compliance requirements that impact PowerPlus’s information security approach. 
%This can include compliance with industry standards, data privacy laws (such as GDPR if applicable), and any sector-specific requirements.

PowerPlus is a multinational energy company, focusing on managing critical infrastructure across Europe and the Americas. With 15,000 employees (including 5,000 external contractors), PowerPlus serves 20 million electricity and 1.3 million gas customers. Its strategic goals emphasize sustainability, innovation, expanding into new regions, boosting renewable energy, and advancing digital transformation.
Operating within a highly regulated industry, PowerPlus complies with stringent sectoral and data privacy regulations, such as GDPR. The Information Security department, part of Corporate IT, works closely with a Corporate Risk division. Their infrastructure spans both Operational Technology (OT) and Information Technology (IT), necessitating tailored security protocols for each.
A key component of PowerPlus's security is its Security Operations Center (SOC), which provides real-time monitoring, incident response, and vulnerability management. The IT environment includes 200 applications, cloud services, extensive data centers for disaster recovery, and a BYOD policy. These measures highlight PowerPlus's commitment to secure operations across a complex technological landscape.

\subsection{Industry-Specific Security Challenges}

%Briefly discuss common security challenges within PowerPlus’s industry, including frequent cyber threats or unique vulnerabilities that similar companies face. This section helps frame the risk assessment within an industry-relevant context, underscoring the necessity for strong cybersecurity measures.

Energy companies like PowerPlus face significant security challenges, including advanced persistent threats (APTs) from nation-state actors, ransomware attacks targeting critical systems, and vulnerabilities in their supply chains. Operational Technology (OT) security is especially crucial, as OT systems often lack the robust protections of IT networks. Insider threats also pose risks, alongside the pressures of meeting regulatory compliance requirements, such as GDPR. These factors highlight the need for strong, layered cybersecurity measures to protect against frequent and evolving threats in the industry.

\section{Objectives of the Risk Assessment and Incident Response Plan}

% List the primary objectives of the risk assessment and incident response plan, focusing on:
% Identifying, analyzing, and evaluating cybersecurity risks to PowerPlus.
% Developing a proactive response strategy for potential security incidents.
% Ensuring alignment with ISO/IEC 27005:2022 standards for systematic and effective risk management.
% Establishing guidelines for PowerPlus’s Security Operations Center (SOC) to detect, respond to, and mitigate incidents.
% These objectives reflect PowerPlus’s commitment to safeguarding its information assets and maintaining business continuity.

This report outlines the key objectives of PowerPlus’s risk assessment and incident response plan, developed to strengthen the organization’s cybersecurity posture and ensure resilient business operations.

\begin{itemize}
    \item \textbf{Cybersecurity Risk Identification and Evaluation:} The primary goal is to systematically identify, analyze, and assess cybersecurity risks affecting PowerPlus, focusing on critical infrastructure and identifying potential vulnerabilities within both IT and Operational Technology (OT) systems.
    \item \textbf{Proactive Incident Response Strategy:} PowerPlus aims to establish a proactive approach to cybersecurity incidents. By preparing strategies to respond effectively to potential threats, PowerPlus can minimize damage and facilitate quick recovery in the event of an incident.
    \item \textbf{Compliance with ISO/IEC 27005:2022 Standards:} The risk management framework is aligned with the ISO/IEC 27005:2022 standards, ensuring that PowerPlus follows systematic and industry-recognized practices in identifying and managing security risks.
    \item \textbf{Guidelines for the Security Operations Center (SOC):} Clear guidelines are set for the SOC to enable prompt detection, response, and mitigation of cybersecurity incidents. The SOC’s protocols include real-time monitoring, analysis, and structured incident response to reduce the impact of threats.
\end{itemize}

\section{Methodology Overview}

% Provide a brief summary of the methodology that will be used in the report. This includes:
% Utilizing ISO/IEC 27005:2022 as the primary standard for risk assessment, with references to ISO/IEC 27001:2022 and ISO 31000:2018 for additional security and risk management guidance.
% Employing a structured approach that involves risk identification, analysis, evaluation, and treatment, as outlined in the standards.
% Developing incident scenarios and SOC preparation strategies based on observed threats and vulnerabilities.

This report will employ a structured methodology for risk assessment and incident response planning, grounded in industry-standard frameworks. The approach is as follows:

\begin{itemize}
    \item \textbf{Primary Standard – ISO/IEC 27005:2022:} ISO/IEC 27005:2022 will serve as the main standard for guiding the risk assessment process, offering a comprehensive framework for systematically identifying, analyzing, evaluating, and treating cybersecurity risks specific to PowerPlus’s infrastructure.
    \item \textbf{Supplementary Standards – ISO/IEC 27001:2022 and ISO 31000:2018:} To enhance the security and risk management framework, references will be made to ISO/IEC 27001:2022 for information security standards and ISO 31000:2018 for general risk management guidance, ensuring a holistic approach to risk and security.
    \item \textbf{Structured Risk Management Process:} The methodology will involve clearly defined phases of risk management: identifying potential risks, analyzing their likelihood and impact, evaluating them in the context of PowerPlus’s operations, and developing appropriate treatments to mitigate or manage identified risks effectively.
    \item \textbf{Incident Scenarios and SOC Strategies:} Based on identified threats and vulnerabilities, specific incident scenarios will be developed. These scenarios will inform the preparation and response strategies for the Security Operations Center (SOC), ensuring that PowerPlus is equipped to handle various potential cybersecurity incidents with proactive and targeted responses.
\end{itemize}

\section{Structure of the Report}

% Summarize the key chapters that follow in this report:
% Establishing the Context: Outlines the scope, assets, risk environment, and methodology applied in the assessment.
% Risk Assessment: Details the process and results of identifying and evaluating risks specific to PowerPlus.
% Incident Scenarios: Conceptualizes three hypothetical attack scenarios to assess PowerPlus’s readiness.
% SOC Preparation: Describes monitoring requirements and three detailed incident response cases for PowerPlus’s SOC.
% Conclusions: Summarizes key findings and recommendations based on the assessment.
% References: Lists all resources and standards referenced throughout the report in APA format.

\begin{itemize}
    \item \textbf{Establishing the Context:} Defines the scope of the assessment, identifies PowerPlus's critical assets, outlines the risk environment, and describes the methodology used for evaluating cybersecurity risks.
    \item \textbf{Risk Assessment:} Documents the process and outcomes of identifying and evaluating risks that are specific to PowerPlus, providing insight into the organization’s security landscape.
    \item \textbf{Incident Scenarios:} Presents three hypothetical attack scenarios to assess PowerPlus’s readiness and resilience, helping to evaluate the effectiveness of current security measures.
    \item \textbf{SOC Preparation:} Details the monitoring requirements for PowerPlus’s Security Operations Center (SOC) and provides three specific incident response cases, offering guidelines for effective response and mitigation.
    \item \textbf{Conclusions:} Summarizes the assessment’s main findings and offers recommendations to enhance PowerPlus’s cybersecurity posture.
    \item \textbf{References:} Lists all standards, frameworks, and resources cited in the report in APA format, ensuring proper attribution and traceability.
\end{itemize}
\let\cleardoublepage\clearpage
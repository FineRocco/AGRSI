\chapter{Risk Assessment}

% Present the results of the various stages of risk assessment based on ISO/IEC 27005:2022.
% Describe the methodology used for:
% - Risk identification
% - Risk analysis
% - Risk evaluation
% Provide a table or list of the eight most important risks and potential controls.

\section{Risk Identification}
% Use ISO/IEC 27005:2022 concepts to explain how risks were identified.

\section{Risk Analysis}
% Explain the analysis of identified risks, including their impact and likelihood.

\section{Risk Evaluation}
% Rank the identified risks and justify their importance.
% Use concrete controls if appropriate to mitigate the risks.

% Example of a table for risk assessment:
% \begin{table}[h]
% \centering
% \begin{tabular}{|l|l|l|l|}
% \hline
% \textbf{Risk ID} & \textbf{Description} & \textbf{Impact} & \textbf{Likelihood} \\
% \hline
% R1 & Example Risk & High & Medium \\
% \hline
% \end{tabular}
% \caption{Risk Assessment Summary}
% \end{table}
\chapter{Risk Assessment}

\section{Risk identification}

Risk identification is a critical component of the overall risk assessment process. The following eight risks are identified as the most significant for PowerPlus, based on its organizational and technological context, in alignment with ISO/IEC 27005:2022:

\begin{enumerate}
    \item \textbf{Data Breaches Affecting Personal Data}
    \begin{itemize}
        \item \textit{Description:} With 20 applications handling 6 million customer records and the use of cloud services (SaaS and IaaS), there is a high risk of data breaches compromising customer data confidentiality.
        \item \textit{Proposed Control:} Implement robust encryption, multi-factor authentication (MFA), and regular audits of data access policies.
        \item \textit{Risk Owner:} Data Protection Officer (DPO) or Information Security Manager.
        \item \textit{Reason:} Responsible for overseeing compliance with data protection regulations and managing risks related to customer data confidentiality.
    \end{itemize}

    \item \textbf{Unsecured Remote Maintenance for OT Systems}
    \begin{itemize}
        \item \textit{Description:} Operational Technology (OT) systems rely on remote maintenance through vendor-contracted services, exposing them to potential vulnerabilities in external access points.
        \item \textit{Proposed Control:} Enforce strict VPN configurations, role-based access control, and monitor remote maintenance activities with real-time logging.
        \item \textit{Risk Owner:} Head of Operational Technology (OT) Security
        \item \textit{Reason:} Accountable for the security of OT systems and ensuring that remote access by external vendors is managed securely.
    \end{itemize}

    \item \textbf{Inadequate Management of Legacy Systems}
    \begin{itemize}
        \item \textit{Description:} PowerPlus uses legacy SAP technologies over 15 years old, which are prone to vulnerabilities due to outdated components.
        \item \textit{Proposed Control:} Implement a phased modernization plan for legacy systems and deploy virtual patching solutions to mitigate risks in the interim.
        \item \textit{Risk Owner:} IT Infrastructure Manager
        \item \textit{Reason:} Oversees the maintenance and modernization of legacy systems and ensures continuity and security during transitions.
    \end{itemize}

    \item \textbf{BYOD Policy Risks}
    \begin{itemize}
        \item \textit{Description:} The Bring Your Own Device (BYOD) policy, without restrictions on platforms or devices, poses risks of unauthorized access and data leaks.
        \item \textit{Proposed Control:} Deploy Mobile Device Management (MDM) tools and enforce endpoint security measures.
        \item \textit{Risk Owner:} IT Policy and Compliance Manager
        \item \textit{Reason:} Accountable for enforcing IT policies, including BYOD standards, and ensuring endpoint security compliance.
    \end{itemize}

    \item \textbf{Physical and Logical Separation Challenges in Data Centers}
    \begin{itemize}
        \item \textit{Description:} The data centers, despite being distanced 100 km apart, may face risks related to improper disaster recovery configurations or simultaneous environmental hazards.
        \item \textit{Proposed Control:} Regularly test disaster recovery plans, enhance redundancy systems, and improve environmental monitoring.
        \item \textit{Risk Owner:} Data Center Operations Manager
        \item \textit{Reason:} Responsible for ensuring physical and logical security, disaster recovery planning, and minimizing environmental hazards.
    \end{itemize}

    \item \textbf{Insider Threats in Call Center Operations}
    \begin{itemize}
        \item \textit{Description:} Call centers with 300 agents require access to internal systems, increasing the risk of insider threats, including unauthorized data access or misuse.
        \item \textit{Proposed Control:} Implement strict identity and access management (IAM) systems and conduct periodic employee training on security protocols.
        \item \textit{Risk Owner:} Call Center Operations Manager
        \item \textit{Reason:} Accountable for managing access to internal systems by call center employees and mitigating potential insider threats.
    \end{itemize}

    \item \textbf{Complexity in Identity and Access Management (IAM)}
    \begin{itemize}
        \item \textit{Description:} Managing user lifecycles and access for 200 applications poses risks of misconfigurations and unauthorized access.
        \item \textit{Proposed Control:} Centralize IAM with automated provisioning and de-provisioning, coupled with periodic access reviews.
        \item \textit{Risk Owner:} Identity and Access Management Lead
        \item \textit{Reason:} Directly responsible for implementing and maintaining centralized IAM systems and overseeing periodic access reviews.
    \end{itemize}

    \item \textbf{Insufficient Threat Detection Capabilities}
    \begin{itemize}
        \item \textit{Description:} Despite a 24/7 Security Operations Center (SOC) and SIEM tools, correlating events across 100 components might lead to undetected threats due to insufficient resources or system limitations.
        \item \textit{Proposed Control:} Enhance threat detection capabilities by integrating AI-driven analytics and increasing SOC workforce capacity.
        \item \textit{Risk Owner:} Security Operations Center (SOC) Manager
        \item \textit{Reason:} Manages the SOC team and ensures threat detection and response capabilities are effective and aligned with organizational needs.
    \end{itemize}
\end{enumerate}

\section{Risk Analysis}

The following is a hybrid analysis of the identified risks, evaluating their consequences and likelihood using a combination of qualitative and quantitative criteria:

\begin{enumerate}
    \item \textbf{Data Breaches Affecting Personal Data}
    \begin{itemize}
        \item \textbf{Consequence:} 4 - Critical (Quantitative estimate: Potential fines of up to €10M or 2\% of global revenue under GDPR, significant reputational damage, and loss of customer trust).
        \item \textbf{Likelihood:} 3 - Moderate (50\% likely in the next 12 months. May happen once every 5 years).
        \item \textbf{Level of Risk:} High (Combination of critical consequences and moderate likelihood).
    \end{itemize}

    \item \textbf{Unsecured Remote Maintenance for OT Systems}
    \begin{itemize}
        \item \textbf{Consequence:} 5 - Catastrophic (Quantitative estimate: Operational disruptions costing €500K per day, plus regulatory penalties for outages in critical infrastructure).
        \item \textbf{Likelihood:} 4 - Likely (75\% likely in the next 12 months. May happen once every year).
        \item \textbf{Level of Risk:} Very High (Combination of catastrophic consequences and likely likelihood).
    \end{itemize}

    \item \textbf{Inadequate Management of Legacy Systems}
    \begin{itemize}
        \item \textbf{Consequence:} 3 - Serious (Quantitative estimate: Costs to address outages and patch vulnerabilities could reach €200K annually, along with potential productivity losses).
        \item \textbf{Likelihood:} 4 - Likely (75\% likely in the next 12 months. May happen once every year).
        \item \textbf{Level of Risk:} High (Combination of serious consequences and likely likelihood).
    \end{itemize}

    \item \textbf{BYOD Policy Risks}
    \begin{itemize}
        \item \textbf{Consequence:} 2 - Significant (Quantitative estimate: Data leaks or breaches from unsecured devices could cost up to €150K, affecting customer data or intellectual property).
        \item \textbf{Likelihood:} 3 - Moderate (50\% likely in the next 12 months. May happen once every 5 years).
        \item \textbf{Level of Risk:} Medium (Combination of significant consequences and moderate likelihood).
    \end{itemize}

    \item \textbf{Physical and Logical Separation Challenges in Data Centers}
    \begin{itemize}
        \item \textbf{Consequence:} 4 - Critical (Quantitative estimate: A dual data center failure could result in recovery costs exceeding €1M and disrupt services for millions of customers).
        \item \textbf{Likelihood:} 2 - Unlikely (25\% likely in the next 12 months. May happen once every 10 years).
        \item \textbf{Level of Risk:} Medium (Combination of critical consequences and unlikely likelihood).
    \end{itemize}

    \item \textbf{Insider Threats in Call Center Operations}
    \begin{itemize}
        \item \textbf{Consequence:} 3 - Serious (Quantitative estimate: Misuse of sensitive customer data could lead to fines and reputational harm amounting to €200K annually).
        \item \textbf{Likelihood:} 3 - Moderate (50\% likely in the next 12 months. May happen once every 5 years).
        \item \textbf{Level of Risk:} Medium (Combination of serious consequences and moderate likelihood).
    \end{itemize}

    \item \textbf{Complexity in Identity and Access Management (IAM)}
    \begin{itemize}
        \item \textbf{Consequence:} 3 - Serious (Quantitative estimate: Misconfigurations could lead to unauthorized access incidents costing €300K annually).
        \item \textbf{Likelihood:} 4 - Likely (75\% likely in the next 12 months. May happen once every year).
        \item \textbf{Level of Risk:} High (Combination of serious consequences and likely likelihood).
    \end{itemize}

    \item \textbf{Insufficient Threat Detection Capabilities}
    \begin{itemize}
        \item \textbf{Consequence:} 5 - Catastrophic (Quantitative estimate: A significant undetected cyberattack could lead to losses exceeding €1M, including downtime, recovery costs, and fines).
        \item \textbf{Likelihood:} 4 - Likely (75\% likely in the next 12 months. May happen once every year).
        \item \textbf{Level of Risk:} Very High (Combination of catastrophic consequences and likely likelihood).
    \end{itemize}
\end{enumerate}

\section{Risk Evaluation}

\subsection*{Comparing the Results of Risk Analysis with the Risk Criteria}

The purpose of this section is to compare the results of the risk analysis with PowerPlus’s defined risk acceptance criteria. This ensures that identified risks are aligned with the organization’s risk appetite, strategic objectives, and operational constraints.

\begin{itemize}
    \item Risk acceptance criteria, as defined in Section 2.4.2, including thresholds for low, medium, and high risks.
    \item Risks with assigned level values based on the combination of consequences and likelihood.
\end{itemize}

The results of this comparison are summarized as follows:

\begin{itemize}
    \item \textbf{Data Breaches Affecting Personal Data:} Categorized as a \textbf{High Risk}. Requires immediate treatment to prevent regulatory breaches and reputational harm. Suggested actions include strengthening encryption, performing regular audits, and enhancing access controls.
    \item \textbf{Unsecured Remote Maintenance for OT Systems:} Categorized as a \textbf{High Risk}. Immediate treatment is critical to secure remote access and prevent exploitation. Recommended actions include strict VPN configurations, role-based access, and real-time monitoring.
    \item \textbf{Inadequate Management of Legacy Systems:} Categorized as a \textbf{High Risk}. Conditional acceptance is possible while implementing a technology refresh plan. Virtual patching and strict monitoring are essential during the transition.
    \item \textbf{BYOD Policy Risks:} Categorized as a \textbf{Medium Risk}. Accepted based on existing MDM controls and endpoint monitoring but requires periodic review to ensure no escalation.
    \item \textbf{Physical and Logical Separation Challenges in Data Centers:} Categorized as a \textbf{Medium Risk}. Accepted based on the adequacy of current disaster recovery measures and geographical separation.
    \item \textbf{Insider Threats in Call Center Operations:} Categorized as a \textbf{Medium Risk}. Accepted with ongoing monitoring and regular employee training to mitigate insider threats.
    \item \textbf{Complexity in Identity and Access Management (IAM):} Categorized as a \textbf{High Risk}. Requires treatment to reduce misconfiguration and unauthorized access incidents. Centralized IAM implementation is a priority.
    \item \textbf{Insufficient Threat Detection Capabilities:} Categorized as a \textbf{High Risk}. Immediate treatment is critical to improve SOC operations and implement advanced threat detection tools.
\end{itemize}

\subsection{Prioritizing the Analyzed Risks for Risk Treatment}

The risks are prioritized for treatment based on the assessed levels, organizational objectives, and the views of relevant stakeholders.

The prioritization of risks for treatment is determined as follows:
\begin{itemize}
    \item \textbf{High Risks:} Immediate priority for treatment. These risks are escalated to senior management or the risk committee for decision-making and action planning.
    \item \textbf{Medium Risks:} Treated based on a case-by-case analysis of existing controls. Monitoring and periodic reviews are critical to prevent escalation.
    \item \textbf{Low Risks:} Accepted without additional measures but monitored for any changes in conditions or context.
\end{itemize}

A prioritized list of risks is provided below:
\begin{enumerate}
    \item \textbf{Unsecured Remote Maintenance for OT Systems}: High
    \item \textbf{Insufficient Threat Detection Capabilities}: High
    \item \textbf{Data Breaches Affecting Personal Data}: High
    \item \textbf{Inadequate Management of Legacy Systems}: High
    \item \textbf{Complexity in Identity and Access Management (IAM)}: High
    \item \textbf{BYOD Policy Risks}: Medium
    \item \textbf{Insider Threats in Call Center Operations}: Medium
    \item \textbf{Physical and Logical Separation Challenges in Data Centers}: Medium
\end{enumerate}
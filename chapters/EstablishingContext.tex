\chapter{Establishing the Context}

\section{Introduction to PowerPlus and Information Security Context}

%Provide a brief introduction to PowerPlus, including its business environment and general objectives. 
%Emphasize the importance of information security in the context of PowerPlus’s operations. Mention any relevant regulatory or compliance requirements and industry standards applicable to PowerPlus.

\section{Scope of the Risk Assessment}

%Define the boundaries of the risk assessment, specifically what aspects of PowerPlus’s information security infrastructure, processes, and assets will be covered in this study. Clarify what areas are out of scope and provide reasons (e.g., data limitations).

\subsection{Objectives}

%Detail the objectives of this assessment, focusing on identifying potential risks that could impact PowerPlus’s operations, data integrity, and reputation, along with initial steps toward incident response planning.

\subsection{Assets and Resources}

%Identify critical assets and resources at PowerPlus that will be evaluated in the risk assessment. These may include:

%Data assets (e.g., customer data, intellectual property)
%Physical assets (e.g., servers, network devices)
%Personnel and operational capabilities
%Technological infrastructure

\section{Risk Environment and Threat Landscape}

%Analyze the risk environment PowerPlus faces, focusing on current cybersecurity threats and potential vulnerabilities. Describe any known internal and external threats, such as:

%Common cyber threats in the industry (e.g., phishing, ransomware)
%Known vulnerabilities in similar organizations
%Potential insider threats or third-party risks

\subsection{Stakeholders}

%List the key stakeholders within PowerPlus who may be impacted by cybersecurity risks, including executive leadership, IT and SOC teams, customers, and third-party vendors. Describe each stakeholder’s role in risk management and security.

\section{Risk Management Framework and Methodology}

% Describe the risk management framework and methodology that will be applied, with reference to ISO/IEC 27005:2022. Summarize the key stages of the ISO/IEC 27005 risk assessment process that will guide this project:

% Risk Identification: Identifying relevant threats, vulnerabilities, and consequences for PowerPlus.
% Risk Analysis: Assessing the likelihood and impact of identified risks.
% Risk Evaluation: Prioritizing risks based on their potential impact and likelihood.
% Risk Treatment: Outlining preliminary risk treatment options (to be expanded in subsequent sections).

%Explain how ISO/IEC 27001:2022 and ISO 31000:2018 will also be referenced, particularly for guidance on implementing security controls and risk response.

\section{Justification for Approach}

%Provide a justification for using the ISO/IEC standards and explain why these are relevant to PowerPlus’s risk assessment. 
%Mention any limitations in available data and describe how these will be addressed within the assessment scope. Also, discuss any additional methodologies (e.g., Ross et al., 2012) that may supplement the ISO standards to strengthen the assessment.

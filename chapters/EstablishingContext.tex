\chapter{Establishing the Context}

\section{Organizational considerations}

\subsection{Definition and Structure of the Organization}
PowerPlus is defined as a group of entities working collaboratively to achieve its objectives within the energy sector. It includes internal and external teams:

\begin{itemize}
    \item \textbf{Internal Workforce:} 10,000 employees working across corporate IT, operational technology (OT), and various business units.
    \item \textbf{External Workforce:} 5,000 external collaborators, supporting application maintenance, operations, and projects.
\end{itemize}

PowerPlus operates as a multifaceted organization encompassing IT and OT domains, with a central Corporate IT structure managing information security and risk management, and decentralized OT operations supported by individual subsidiaries.

\subsection{Risk Appetite and Governance}
PowerPlus's risk appetite is influenced by its:

\begin{itemize}
    \item \textbf{Size and Complexity:} Operating in 12 countries, the organization serves 20 million electricity customers and 1.3 million gas customers, necessitating a robust risk framework.
    \item \textbf{Sectoral Dynamics:} Operating under stringent energy regulations in Europe and the Americas.
    \item \textbf{Strategic Goals:} Objectives such as digital transformation, renewable energy expansion, and maintaining controlled risks highlight a balanced approach to risk acceptance and mitigation.
\end{itemize}

\subsection{Risk Ownership}

PowerPlus ensures that risk ownership is clearly defined within its governance framework:

\begin{itemize}
    \item \textbf{Accountability and Authority:} Risk owners, primarily at the corporate and business unit levels, are entrusted with the authority and responsibility to manage identified risks.
    \item \textbf{Specialized Departments:} The PP Risk Department and the Security Operations Center (SOC) play pivotal roles in overseeing and mitigating risks related to cybersecurity, operational disruptions, and compliance.
\end{itemize}

\section{Identifying Basic Requirements of Interested Parties}

\subsection*{A) Description of Information Security Controls Adopted by the Organization (PowerPlus) to Ensure Compliance with the ISO/IEC 27001:2022 Standard}

PowerPlus adopts a series of information security controls aligned with the requirements of the ISO/IEC 27001:2022 standard, with a focus on protecting the confidentiality, integrity, and availability of information and organizational assets. The main controls are presented below, considering the organizational structure, technological reality, and challenges faced by PowerPlus.

\subsection*{Organizational Controls:}

\begin{itemize}
    \item \textbf{Information Security Policies (5.1):} PowerPlus has information security policies communicated and understood by all employees and stakeholders. The policies are aligned with corporate objectives and the ISO/IEC 27001:2022 standard.
    \item \textbf{Identity and Access Management (5.16, 5.18):} The company uses strict processes for access control, including the use of identity and access management tools. This involves managing the user lifecycle and assigning permissions in critical systems.
    \item \textbf{Supplier Management (5.19, 5.20):} PowerPlus ensures that information security is monitored across the entire supplier chain, with specific controls for data protection, especially with suppliers who maintain the applications used by the company.
    \item \textbf{Security Incident Management (5.24 to 5.27):} PowerPlus operates a 24/7 \textit{Security Operations Center (SOC)} that monitors cybersecurity incidents in real-time, performing detection, response, and learning from previous incidents, using tools like \textit{SIEM (Security Information and Event Management)}.
\end{itemize}

\subsection*{Human Controls:}

\begin{itemize}
    \item \textbf{Training and Awareness (6.3):} PowerPlus promotes continuous information security training for all employees, ensuring that everyone is up to date with policies and procedures related to security. The \textit{Information Security and IT Risk Management} department coordinates this training.
    \item \textbf{Disciplinary Process (6.4):} A formal disciplinary process is in place to handle security violations, ensuring that infractions are dealt with in an appropriate and transparent manner.
    \item \textbf{Post-Termination Responsibilities (6.5):} When an employee leaves the organization or changes roles, their security responsibilities are maintained and monitored to avoid risks to the organization.
\end{itemize}

\subsection*{Physical Controls:}

\begin{itemize}
    \item \textbf{Physical Security and Perimeters (7.1, 7.2):} PowerPlus adopts stringent physical security measures to protect sensitive areas from unauthorized access. Physical security is managed by one of the companies within the PowerPlus group and integrates appropriate perimeter control.
    \item \textbf{Protection Against Environmental Threats (7.5):} Critical infrastructure, including data centers and OT (Operational Technology) systems, is designed to withstand environmental threats such as natural disasters and ensure business continuity.
\end{itemize}

\subsection*{Technological Controls:}

\begin{itemize}
    \item \textbf{Protection Against Malware (8.7):} PowerPlus implements robust solutions to protect against malware, complemented by continuous awareness and education programs for employees.
    \item \textbf{Technical Vulnerability Management (8.8):} The \textit{Security Operations Center} continuously assesses vulnerabilities and applies necessary corrections to mitigate risks. Penetration tests and forensic analysis are also periodically conducted.
    \item \textbf{Network and System Security (8.20, 8.21):} PowerPlus rigorously manages and monitors its networks and systems, ensuring that communication and data traffic are secure. \textit{Oracle} and \textit{SAP} technologies are widely used within the organization, with intensive access control and monitoring.
\end{itemize}

\subsection*{D) Specific international and/or national regulations}
PowerPlus is subject to strict regulations due to its operations in the energy sector across different regions, including Europe and the American continent. These regulations aim to ensure the security, data privacy, and continuity of operations in a highly critical sector.

\subsection*{E) The organization's internal security rules}
PowerPlus has established internal security rules that cover various aspects of information security, including the following:

\begin{itemize}
    \item Ensures proper control of the user lifecycle and access rights to applications and systems.
    \item  Monitors security incidents 24/7, conducts forensic analysis, manages vulnerabilities, and promotes security awareness among employees.
    \item  Managed through a centralized SIEM (Security Information and Event Management) system, correlating data from more than 100 technological components.
\end{itemize}

\subsection*{F) Security rules and controls from contracts or agreements:}
PowerPlus applies stringent security controls in contracts with suppliers, particularly regarding the maintenance of its applications and protection of personal data. These controls ensure continuity of IT services and compliance with security regulations.

\subsection*{G) Security controls implemented based on previous risk treatment activities:}
The company implements security controls based on prior risk assessments, utilizing the Security Operations Center (SOC) and a SIEM system for monitoring and managing incidents and vulnerabilities. These controls also include identity and access management, ensuring continuous protection of data and systems.


\section{Applying risk assesment}

PowerPlus incorporates risk assessments across various organizational processes to ensure comprehensive risk management. These processes include:

\begin{itemize}
    \item \textbf{Project Management:} Evaluating risks associated with implementing new projects, such as integrating renewable energy solutions or expanding into new geographies.

    \item \textbf{Vulnerability Management:} Regularly assessing technological vulnerabilities, especially within its operational technology (OT) and IT systems, which are managed under distinct domains.

    \item \textbf{Incident Management:} The Security Operations Center (SOC) operates 24/7 to detect, assess, and manage cybersecurity incidents, utilizing SIEM tools to analyze data from over 100 technological components.

    \item \textbf{Problem Management:} Addressing recurring issues, such as vulnerabilities in legacy SAP applications.

    \item \textbf{Impromptu Risk Assessments:} Tackling specific ad-hoc concerns, such as risks associated with BYOD policies or cloud services managing personal data.
\end{itemize}

\section{Establishing and maintaining information security risk criteria}

\subsection{General}
PowerPlus establishes and maintains information security risk criteria based on the requirements of ISO/IEC 27001:2022 ((section 6.1.2.a)) and ISO/IEC 27005:2022. These criteria are designed to ensure that:
\begin{itemize}
    \item Risks are assessed consistently, ensuring reliable and comparable results.
    \item Decisions on risk treatment and acceptance are aligned with strategic objectives and organizational capacity.
\end{itemize}
The criteria used are as follows:
\begin{itemize}
    \item Impact on organizational objectives \- Here the potential costs of an outage (e.g., lost revenue, regulatory fines) are evaluated in financial terms. The reputational impact if the incident happens is also assessed.  In addition, the impact that the risk will have on the continuity of the company's services is analysed and whether we are going against specific rules of the energy sector in this case.
    \item Likelihood \- A historical analysis is made to understand the probability that the risk will have to happen.
    \item CIA \- The potential leakage of sensitive information (confidentiality) is analyzed. In addition to data leakage, it is necessary to understand if the data remains accurate and reliable, so as not to affect integrity. Finally, it is also important to understand the company's availability when suffering a certain attack.
    \item Response time and recovery \- Another important criteria is the time that the sistem that to recover from an incident.
    \item Combination e Risk Sequence \- Analyze how multiple risks can occur simultaneously.
\end{itemize}

\subsection{Risk acceptance criteria}

Risk acceptance criteria sets out the guidelines that PowerPlus uses to determine whether an identified risk is acceptable or requires further treatment. These criteria are key to ensuring consistency in the risk management process, aligning risk acceptance decisions with risk appetite, strategic objectives, and organizational constraints.
In this section, acceptable risk levels, authorities responsible for taking such decisions, conditions for risk acceptance and review and adjustment of criteria will be defined. 

\textbf{Risk Levels}

Low risks have a low probability of occurrence and reduced impact. Generally, these risks do not affect critical systems such as OT and IT, nor regulatory compliance. Because they are considered of little relevance, they can be accepted without the need for additional measures.
Medium risks have a moderate probability or impact. While not critical, they can cause significant disruptions to important processes such as internal IT operations or customer service. These risks are analyzed more closely and generally accepted based on existing controls, but they are regularly monitored to prevent them from evolving to more critical levels.
High risks, on the other hand, have a high probability of occurrence and/or serious impact. These risks can compromise the continuity of critical services, lead to regulatory breaches, or cause the loss of sensitive customer data. Due to their severity, these risks are not directly accepted and require immediate treatment, such as the implementation of additional controls or specific mitigation measures.

\textbf{Risk Acceptance Authority}

At PowerPlus, risk acceptance is distributed according to severity and hierarchical level. Low risks are accepted by operational managers, while medium risks require approval from directors, especially if they affect strategic systems. High risks, which can compromise critical systems or violate regulations, are analyzed by senior management or the risk committee. This structure ensures decisions aligned with the organization's strategic objectives and risk appetite.

\textbf{Conditional acceptance}

For example, a risk associated with a legacy system may be temporarily accepted while a technology refresh plan is being executed. Similarly, conditional acceptance can be applied in scenarios where a high risk is unavoidable, but mitigation strategies such as contingency plans or enhanced monitoring are in place to minimize potential impacts.
These conditions ensure flexibility in risk management, without compromising PowerPlus' security or regulatory compliance.

\textbf{Review and Adjustment of Criteria}

PowerPlus' risk acceptance criteria will be reviewed annually to ensure that they remain aligned with changes in the organizational context, such as technological upgrades, regulatory changes, or changes in strategic objectives. This includes an ongoing analysis of factors such as new cyber threats, modifications to business processes, and the evolution of IT and OT infrastructure.

The review of the criteria also considers feedback from security incidents or resilience tests carried out by the organization, adjusting the acceptance criteria as necessary. With this, PowerPlus ensures that its risk acceptance criteria remain effective, ensuring the continuous protection of critical assets and compliance with legal obligations.


\subsection{Criteria for performing information security risk assessments}

\subsubsection{General}

Criteria for performing information security risk assessments is essential to define clear and consistent criteria for the assessment of information security risks, ensuring that risks are analyzed effectively. These criteria aim to determine the significance of risks in terms of their consequences, probability of occurrence and level of risk. The definition of such criteria must consider factors such as the classification of information, the quantity and concentration of data, the strategic importance of the business processes involved, as well as the operational criticism of the availability, confidentiality, and integrity of information.

\subsubsection{Consequence Criteria}

Here are the consequence criteria for PowerPlus, taking into account the company's context in the renewable energy sector and its technological operations:

\begin{itemize}
    \item \textbf{Minor:} Negligible consequences for PowerPlus. \\
    No significant operational impact on energy generation activities and no risks to the safety of people or property. \\
    \textit{Example:} A minor failure in a secondary system that consumes operational margins but does not affect the organization’s objectives.
    
    \item \textbf{Significant:} Limited but relevant consequences for PowerPlus. \\
    Partial degradation of activities, such as temporary reductions in energy generation efficiency, without endangering the safety of people or assets. \\
    \textit{Example:} A failure in a single wind turbine causing reduced output but allowing continued operation in a degraded mode.
    
    \item \textbf{Serious:} Substantial consequences for PowerPlus. \\
    Significant operational degradation with potential implications for the safety of people or assets. \\
    \textit{Example:} Simultaneous failures in several energy assets resulting in a disruption of a significant portion of energy supply, creating severe operational challenges but no direct impact on the energy sector as a whole.
    
    \item \textbf{Critical:} Disastrous consequences for PowerPlus. \\
    Inability to maintain essential activities, with serious consequences for the safety of people or property. \\
    \textit{Example:} A catastrophic failure in a solar plant leading to fires or significant damage, threatening the company’s continuity and possibly impacting sectors reliant on the energy provided.
    
    \item \textbf{Catastrophic:} Consequences beyond PowerPlus, affecting the energy sector or society at large. \\
    Substantial impact on the renewable energy sector, with potential regulatory or environmental implications. \\
    \textit{Example:} A major incident involving the company’s infrastructure, causing severe environmental pollution or widespread energy outages, affecting critical sectors and requiring governmental response.
\end{itemize}


\subsubsection{Likelihood Criteria}

In this section on Likelihood criteria, the objective is to define how the probability of a risk occurring will be assessed. That is, we need to specify how the company will determine the chance of a risk occurring based on different factors. To do that it is necessary to specify the factors that can influence the probability and to show how a probability is going to be measure, using a scale.

\textbf{Factors that can influence the probability}

\begin{itemize}
    \item Accidental or natural events \- Natural disasters or industrial accidents can increase the likelihood of service disruptions.
    \item Exposure of information or assets \- Exposure of critical data or systems to the internet or threats increases risk.
    \item Exploitable vulnerabilities \- Security flaws or outdated systems make risks more likely.
    \item Technology failures \- Hardware or software problems, common in power systems, increase the risk of operational failures.
    \item Human errors \- Failures due to negligence or lack of training can increase the likelihood of incidents.    
\end{itemize}

\textbf{Probability measurement}

To measure the likelihood of risks at PowerPlus, the following qualitative scale will be used:

\begin{itemize}
    \item \textbf{0 - Not Applicable:} 0\% likelihood in the next 12 months. \textit{(Will never happen)}.
    \item \textbf{1 - Rare:} 5\% likelihood in the next 12 months. \textit{(May happen once every 20 years)}.
    \item \textbf{2 - Unlikely:} 25\% likelihood in the next 12 months. \textit{(May happen once every 10 years)}.
    \item \textbf{3 - Moderate:} 50\% likelihood in the next 12 months. \textit{(May happen once every 5 years)}.
    \item \textbf{4 - Likely:} 75\% likelihood in the next 12 months. \textit{(May happen once every year)}.
    \item \textbf{5 - Almost Certain:} 100\% likelihood in the next 12 months. \textit{(May happen multiple times a year)}.
\end{itemize}

To determine the chance of a risk occurring, we can look at how often similar events have occurred in the past. If, for example, technological failures or operational failures have already occurred several times, the chance of occurrence increases.

\subsubsection{Criteria for determining the level of risk}

The level of risk is determined by analyzing the combination of the likelihood of an event occurring and the severity of its consequences. The following matrix will guide the risk evaluation process:

\begin{table}[h!]
\centering
\begin{tabular}{|l|l|l|l|l|l|l|}
\hline
\textbf{Likelihood}   & \textbf{Catastrophic} & \textbf{Critical} & \textbf{Serious} & \textbf{Significant} & \textbf{Minor} \\ \hline
\textbf{Almost certain} & Very high           & Very high          & High             & High                 & Medium         \\ \hline
\textbf{Likely}         & Very high           & High               & High             & Medium               & Low            \\ \hline
\textbf{Moderate}       & High                & High               & Medium           & Low                  & Low            \\ \hline
\textbf{Unlikely}       & Medium              & Medium             & Low              & Low                  & Very low       \\ \hline
\textbf{Rare}           & Low                 & Low                & Low              & Very low             & Very low       \\ \hline
\textbf{Not applicable} & Very low            & Very low           & Very low         & Very low             & Very low       \\ \hline
\end{tabular}
\caption{Risk Level Matrix for PowerPlus}
\end{table}

\textbf{Explanation of the Risk Matrix:}
\begin{itemize}
    \item \textbf{Likelihood Levels:} The likelihood of an event is categorized into six levels: Almost certain, Likely, Moderate, Unlikely, Rare, and Not applicable, as defined earlier.
    \item \textbf{Consequence Levels:} Consequences are categorized from Catastrophic to Minor, as defined in the consequence criteria.
    \item \textbf{Risk Ratings:} The intersection of likelihood and consequence determines the risk rating (e.g., Very High, High, Medium, Low, Very Low).
    \item \textbf{Purpose:} This matrix helps prioritize risks, ensuring that resources are allocated to mitigate the most critical risks effectively.
\end{itemize}

\section{Choosing an Appropriate Method for Information Security Risk Assessment at PowerPlus}

In the context of PowerPlus, the selection of a method for managing information security risks must align with the organization's overall risk management approach, as defined in \textbf{ISO/IEC 27001:2022, 6.1.2 b)}. This method must ensure that risk assessments are \textbf{consistent}, \textbf{valid}, and \textbf{comparable}.

\subsection{Consistency}

To ensure consistency, risk assessments should follow a standardized process where the same risks, when evaluated by different individuals or at different times, yield similar results.  
At PowerPlus, this can be achieved through tools like \textbf{SIEM (Security Information and Event Management)}, which correlates data from various sources for continuous and consistent monitoring of security incidents.

\subsection{ Comparability}

PowerPlus should define clear and uniform risk assessment criteria to enable objective comparisons between different risks.  
A \textbf{risk matrix} based on impact and likelihood can be used to classify risks in both IT (Information Technology) and OT (Operational Technology), ensuring that decisions are aligned with business strategy.

\subsection{ Validity}

The validity of risk assessments depends on their adherence to operational reality. This requires using \textbf{real-world scenarios} and \textbf{updated data} on vulnerabilities and impacts.  
PowerPlus should consider:
\begin{itemize}
    \item \textbf{200 corporate applications} handling critical data;
    \item \textbf{6 million personal data records};
    \item \textbf{Cloud services (SaaS and IaaS)}, which expand the attack surface.
\end{itemize}
Periodic assessments and audits will help continuously validate the results, ensuring compliance with changes in the threat landscape.

\vspace{.7cm}
We conclude that adopting a risk management method that ensures \textbf{consistency}, \textbf{comparability}, and \textbf{validity} will enable PowerPlus to meet the requirements of \textbf{ISO/IEC 27001:2022} and strengthen its information security, supporting its objectives of growth, innovation, and risk control.


